\documentclass{beamer}
\title{A talk on the RSA Cryptosystem}
\author{\texttt{alex@corcoles.net}}
\date{Barcelona Compsci Club}

\begin{document}

\begin{frame}
	\titlepage
\end{frame}

\begin{frame}
	\frametitle{Outline}
	\tableofcontents
\end{frame}

\section{Why}

\begin{frame}
	\frametitle{Why am I doing this?}
	\begin{itemize}
		\item We need speakers!
		\item RSA is important and interesting
		\item Need to practice, learn Beamer
	\end{itemize}
\end{frame}

\section{Before RSA}

\begin{frame}
	\frametitle{Life before RSA}
	\begin{itemize}
		\item We had symmetric key algorithms
		\begin{itemize}
			\item Hebrew Atbash cipher -- 600 BC
			\item Caesar cipher -- 40 BC
			\item Enigma -- 1920
			\item DES -- 1975
		\end{itemize}
		\item RSA precursors (key exchange)
		\begin{itemize}
			\item Merkle's Puzzles -- 1974
			\item Diffie-Hellman -- 1976
		\end{itemize}
		
	\end{itemize}
\end{frame}

\begin{frame}
	\frametitle{The seventies in computing}
	\begin{itemize}
		\item 1970 -- 4-node ARPANET
		\item 1972 -- Intel 8008 -- first 8-bit Intel
		\item 1973 -- Ethernet
		\item 1976 -- Cray I -- 4Mb RAM, Apple I
		\item 1978 -- VAX 11/780 -- 4Gb addressable RAM
	\end{itemize}
\end{frame}

\section{The Paper}

\begin{frame}
	\frametitle{The Paper}
		A Method for Obtaining Digital Signatures and Public-Key Cryptosystems \\
		R.L. Rivest, A. Shamir, and L. Adleman -- April 4, 1977
		\begin{itemize}
			\item Public key, security and signatures
			\item Method
			\item Mathematical foundations
			\item Efficient implementation
			\item Security
		\end{itemize}
\end{frame}

\subsection{Mathematical foundations}

\begin{frame}
	\frametitle{Mathematical foundations}
	\begin{itemize}
		\item Discrete math
		\item Euler's totient function $\varphi$
		\item Fermat's Little Theorem or Euler's Theorem
		\item Large prime numbers are easy to find (and large
		      composites are hard to decompose)
		\item Some optimizations
	\end{itemize}

	Much of this is number theory, which didn't have practical
        uses before
\end{frame}

\begin{frame}
	\frametitle{A note about modulus}

	Modulus (mod in math, \% in many programming languages)

	$$ 7 \mod 3 = 1 $$
	$$ 11 \mod 6 = 5 $$

	$$ 1 \equiv 4 \equiv 7 \pmod{3} $$
	$$ 5 \equiv 11 \equiv 17 \pmod{6} $$
\end{frame}

\subsection{How it works}

\begin{frame}
	\frametitle{How it works, some mathematical gibberish}
	\begin{itemize}
		\item Two large prime numbers $p$ and $q$ (large primes are
                      easy to find)
		\item $n = pq$
		\item $\varphi(n) = \varphi(p)\varphi(q) = (p-1)(q-1) = n - (p + q -1)$
		      (Euler's totient)
		\item $e$ such that $1 < e < \varphi(n)$ and $\gcd(e, \varphi(n)) = 1$
		\item Find $d$ such that $d \equiv e^{-1} \pmod{\varphi(n)}$
		      (discrete math, modulo arithmetic)
	\end{itemize}

	\begin{itemize}
		\item Public key: $n$, $e$
		\item Private key $n$, $d$ and everything else
	\end{itemize}
\end{frame}

\begin{frame}
	\frametitle{Encryption and decryption functions}
	\begin{itemize}
		\item $m$ message
		\item $c \equiv m^e \pmod{n}$ encryption
		\item $m \equiv c^d \pmod{n}$ decryption
		\item ($m^e)^d \equiv m \pmod{n}$ proved by Euler or Fermat
		\item Need `optimizations' to calculate large exponents
		      `quickly'
	\end{itemize}
\end{frame}

\subsection{How can it be used}

\begin{frame}
	\frametitle{How it can be used}

	\begin{itemize}
		\item You publish your public key, keep your private key
		      private

		\item Anyone can encrypt a message with your public key,
		      only you can decrypt it with your private key

		\item You can send out a message along with the message
		      encoded with your private key. Anyone can decode the
		      encoded message with your public key and verify
		      it's equal to the unencoded message
	\end{itemize}
\end{frame}

\section{Impact}

\begin{frame}
	\frametitle{Impact}

	\begin{itemize}
		\item Public key encryption is very practical (Internet
		      communications, for instance)
		\item Signature didn't even exist
		\item Symmetric is much faster than asymmetric, though
		\item Hardy will be pissed:
	\end{itemize}

	\begin{exampleblock}{}
		``No one has yet discovered any warlike purpose to be served
		by the theory of numbers or relativity, and it seems
		unlikely that anyone will do so for many years''
	\end{exampleblock}
\end{frame}

\section{The twist}

\begin{frame}
	\frametitle{British spies}

	\begin{itemize}
		\item 1970 -- James H. Ellis (GHCQ\footnote{Government
		      Communications Headquarters}, formerly GC \&
		      CS\footnote{Government Code and Cypher School}, where
		      Alan Turing worked) proves that asymmetric encryption is
		      possible
		\item 1974 -- Clifford Cocks, basically finds out the
		      operations that work in Ellis' paper and is equivalent
		      to RSA (3 years before)
		\item 1976 -- Malcolm Williamson invents Diffie-Helman (a
		      few months before)
	\end{itemize}

	RSA is just \ldots intuitive? obvious? Everything fits together
\end{frame}

\section{Related papers}

\begin{frame}
	\frametitle{Related papers}

	\begin{itemize}
		\item Diffie, W.; Hellman, M. (1976). "New directions in
		      cryptography". IEEE Transactions on Information Theory 22 
		      (6): 644–654. doi:10.1109/TIT.1976.1055638
		\item The History of Non-Secret Encryption JH Ellis 1987
	\end{itemize}
\end{frame}

\section{The end}

\begin{frame}
	\frametitle{The end}

	\begin{itemize}
		\item Thank you for enduring this
		\item Questions?
	\end{itemize}
\end{frame}

\end{document}
