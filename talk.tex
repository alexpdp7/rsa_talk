\documentclass{beamer}
\title{A talk on the RSA Cryptosystem}
\author{\texttt{alex@corcoles.net}}
\date{Barcelona Compsci Club}

\begin{document}

\begin{frame}
	\titlepage
\end{frame}

\begin{frame}
	\frametitle{Outline}
	\tableofcontents
\end{frame}

\section{Why}

\begin{frame}
	\frametitle{Why am I doing this?}
	\begin{itemize}
		\item We need speakers!
		\item RSA is important and interesting
		\item Need to practice, learn Beamer
	\end{itemize}
\end{frame}

\section{Before RSA}

\begin{frame}
	\frametitle{Life before RSA}
	\begin{itemize}
		\item We had symmetric key algorithms
		\begin{itemize}
			\item Hieroglyphs -- 1900 BC
			\item Mesopotamia Clay Tablets -- 1500 BC
			\item Hebrew Atbash cipher -- 600 BC
			\item Caesar cipher -- 40 BC
			\item Enigma -- 1920
			\item DES -- 1975
		\end{itemize}
	\end{itemize}
\end{frame}

\begin{frame}
	\frametitle{1970}
	\begin{itemize}
		\item 1970 -- 4-node ARPANET
		\item 1972 -- Intel 8008 -- first 8-bit Intel
		\item 1973 -- Ethernet
		\item 1976 -- Cray I -- 4Mb RAM
	\end{itemize}
\end{frame}

\section{The Paper}

\begin{frame}
	\frametitle{The Paper}
		A Method for Obtaining Digital Signatures and Public-Key Cryptosystems \\
		R.L. Rivest, A. Shamir, and L. Adleman -- April 4, 1977
		\begin{itemize}
			\item Public key, security and signatures
			\item Method
			\item Mathematical foundations
			\item Efficient implementation
			\item Security
			\item \ldots
			\item Self-contained
			\item Practical
		\end{itemize}
\end{frame}

\begin{frame}
	\frametitle{Mathematical foundations}
	\begin{itemize}
		\item Discrete math
		\item Euler's totient function $\varphi$
		\item Fermat's Little Theorem or Euler's Theorem
		\item Large prime numbers are easy to find (and large
		      composites are hard to decompose)
		\item Some optimizations
	\end{itemize}
\end{frame}

\subsection{How it works}

\begin{frame}
	\frametitle{How it works}
	\begin{itemize}
		\item Two large prime numbers $p$ and $q$ (large primes are
                      easy to find)
		\item $n = pq$
		\item $\varphi(n) = \varphi(p)\varphi(q) = (p-1)(q-1) = n - (p + q -1)$
		      (Euler's totient)
		\item $e$ such that $1 < e < \varphi(n)$ and $\gcd(e, \varphi(n)) = 1$
		\item Find $d$ such that $d \equiv e^{-1} \pmod{\varphi(n)}$
		      (discrete math, modulo arithmetic)
		\item Public key: $n$, $e$ Private key $n$, $d$ and everything else
	\end{itemize}
\end{frame}

\begin{frame}
	\frametitle{Encryption and decryption functions}
	\begin{itemize}
		\item $m$ message
		\item $c \equiv m^e \pmod{n}$ encryption
		\item $m \equiv c^d \pmod{n}$ decryption
		\item ($m^e)^d \equiv m \pmod{n}$ proved by Euler or Fermat
		\item Need `optimizations' to calculate large exponents
		      `quickly'
	\end{itemize}
\end{frame}

\end{document}
